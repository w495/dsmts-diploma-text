%%%%%%%%%%%%%%%%%%%%%%%%%%%%%%%%%%%%%%%%%%%%%%%%%%%%%%%%%%%%%%%%%%%%%%%%%%%%%%%%
%%%
%%% ABSTRACT
%%%

\section*{Реферат}

	Дипломная работа содержит 
		$104$~страницы, 
		$13$~рисунков, 
		$21$~таблицу,
		$4$~приложения.
	Список использованных источников содержит 
		$63$ позиции.\\ \\
		% \total{theappsection} 
		%	...
		% \total{citnum} 
		%%%
		%%% Тут очень хочется использовать totcount.
		%%% Но он не учитывает склонения самостоятельно, 
		%%% 	потому проще такие вещи прописывать руками,
		%%% 	благо они обычно меняются не часто.
		%%% 
		%%% В преамбуле:
		%%% ---------------------------------------------
		%%%	\usepackage{totcount}
		%%%	\regtotcounter{page}
		%%%	\regtotcounter{figure}
		%%%	\regtotcounter{table}
		%%%	\regtotcounter{theappsection}
		%%%	\newtotcounter{citnum}
		%%%	\def\oldbibitem{} \let\oldbibitem=\bibitem
		%%%	\def\bibitem{\stepcounter{citnum}\oldbibitem}
		%%%	\newtotcounter{citesnum}
		%%%	\def\oldcite{} \let\oldcite=\cite
		%%%	\def\cite{\stepcounter{citesnum}\oldcite}
		%%% ---------------------------------------------
		%%% 
	Ключевые слова: \\ 
	\uppercase{
		\worktextkeywords
	}\\ \\
	Работа посвящена разработке распределенной статистической системы 
	перевода естественных языков.
	Актуальность темы оправдана появлением большого количества научно-технических 
	документов и~необходимостью оперативного их перевода на другие языки. 
	В работе проведен краткий обзор существующих типов систем машинного перевода,
	описана теоретическая база статистических систем машинного перевода, 
	изложен нетрадиционный подход к~созданию таких систем. 
	В результате работы было создано распределенное 
	программно-информационное обеспечение
	статистической модели перевода научно-технических текстов 
	на~примере русского и~английского языков.
	Система представляет набор приложений 
	взаимодействующих с~общей базой данных.
	Набор приложений можно разделить на~два класса:
	\begin{ditemize}
		\item приложения необходимые для обучения системы по~уже~имеющимся
		переводам, которые выполнены человеком;
		\item приложения осуществляющие подбор наиболее 
		подходящих переводных эквивалентов.
	\end{ditemize}
	Алгоритмы обучения системы были разработаны c учетом 
	особенностей научных текстов
	и слабо применимы для других стилей литературы.
	За~неимением текстов нужного объема и~качества,
	в рамках данной работы обучение системы проводилось
	на комбинированном наборе переводов, состоящим 
	преимущественно
	из официально-делового и~публицистического стилей литературы.
	Для подбора наиболее подходящих переводных эквивалентов
	используется жадный инкрементный поиск.
	Его основным преимуществом является высокая скорость работы,
	что может оказаться важным для оперативного перевода.
	Качество перевода разработанной системы несколько уступает
	существующим аналогам. Это объясняется особенностями
	исходных данных и~характером используемых алгоритмов.
	Скорость работы системы в несколько раз превосходит 
	скорости доступных систем подобного 
	класса. Для~сравнения систем использовался 
	одинаковый набор данных.
	В~экономической части проведен расчет 
	стоимости разработанной стемы.
	В~разделе посвященном 
	охране труда и~окружающей среды описано 
	каких последствий можно избежать 
	при использовании созданной системы.

\pagebreak

