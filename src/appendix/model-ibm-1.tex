%%%%%%%%%%%%%%%%%%%%%%%%%%%%%%%%%%%%%%%%%%%%%%%%%%%%%%%%%%%%%%%%%%%%%%%%%%%%%%%%
%%%
%%% Приложение
%%%


\subsection{Модель IBM 1}

\newcommand{\Com}[1]{{\color[rgb]{0,0.5,0} \itshape \ \Comment #1 }}

{ % \footnotesize\sffamily
\begin{codebox}
	\Procname{Обучить-Модель-IBM-1 ($t(\WE|\WR)$, $\TE$, $\TR$)}
	\li 	$\forall \ \WE \in \SE \in \TE $ \Do 	
	\li 		$\forall \WR \in \SR \in \TR $ \Do 
	\li 			$ t(\WE|\WR) \gets u, u \in \mathbb{R}$;
				\End	
			\End	
	\li 	\Com{ Инициализируем таблицу $t(\WE|\WR)$ одинаковыми значениями.}
	\li 	\Пока {не сойдется} \Do 
	\li 		$\forall \ \WE \in \SE \in \TE $ \Do \Com{ Инициализируем остальные таблицы.}
	\li 			$\forall \WR \in \SR \in \TR $ \Do 
	\li 				$ counts(\WE|\WR) \gets  0$;  $\quad total(\WR) \gets  0$;
					\End	
				\End	
	\li 		$\forall \ \SE, \SR \in \TE, \TR$ \Do \Com{ Вычисляем нормализациию. }
	\li 			$\forall \ \WE \in \SE$ \Do
	\li 				$stotal(\WE) \gets  0$;
	\li 				$\forall \ \WR \in \SR$ \Do	 
	\li 					$stotal(\WE) \gets stotal(\WE) + t(\WE|\WR)$;
	 					\End	
	 				\End	
	\li 			$\forall \ \WE \in \SE$ \Do \Com{ Собираем подсчеты. }
	\li 				$\forall \ \WR \in \SR$ \Do	 
	\li 					$counts(\WE|\WR) \gets counts(\WE|\WR) + \dfrac{t(\WE|\WR)}{stotal(\WE)}$;
	\li 					$total(\WR) \gets total(\WR) + \dfrac{t(\WE|\WR)}{stotal(\WE)}$;
		 				\End	
					\End
				\End
	\li 		$\forall \ \WE \in \SE \in \TE $ \Do \Com{ Оцениваем вероятность.}
	\li 			$\forall \WR \in \SR \in \TR $ \Do 
	\li 				$ t(\WE|\WR) \gets \dfrac{counts(\WE|\WR)}{total(\WR)}$;
					\End	
				\End	
	\li		\End
\end{codebox}
}


\pagebreak

