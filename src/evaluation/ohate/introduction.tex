%%%%%%%%%%%%%%%%%%%%%%%%%%%%%%%%%%%%%%%%%%%%%%%%%%%%%%%%%%%%%%%%%%%%%%%%%%%%%%%%
%%%
%%% Введение
%%%

\subsection{Введение}

Любая технологическая деятельность является потенциально опасной \\
для~человека и~окружающей среды.
Разработка дипломного проекта на~кафедре <<Вычислительная математика  и~программирование>> 
неизбежно связано с~постоянной и~долговременной 
работой с~персональным вычислительным устройством 
и иной электронной техникой. 

Распределенное программно-информационное обеспечение 
статистической модели перевода естественных языков,
созданное в рамках данной дипломной работы, 
--- это сложный программный комплекс.
Разработка такого рода систем подразумевает 
постоянное взаимодействие с~компьютерной техникой.
Работа с~компьютером характеризуется:
\begin{itemize}
	\item  значительным умственным напряжением;
	\item  нервно-эмоциональной нагрузкой;
	\item высокой напряженностью зрительной работы;
	\item неестественным положением корпуса тела;
	\item нагрузкой на~мышцы кистей рук (при работе с~клавиатурой). 
\end{itemize}
Большое значение имеет рациональная конструкция и~расположение элементов рабочего места, 
что~важно для поддержания оптимальной рабочей позы человека.
В процессе работы с~компьютером необходимо соблюдать правильный режим труда и~отдыха. 
В противном случае у человека отмечаются значительное
напряжение зрительного аппарата с~появлением жалоб на~неудовлетворенность
работой, головные боли, раздражительность, нарушение сна, усталость 
и~болезненные ощущения в глазах, в пояснице, в области шеи и~руках.
Кроме того, важно соблюдать достаточную площадь на~одно рабочее место. 
Она должна составлять для взрослых пользователей не менее 9 м$^2$, 
а~объем не менее 20 м$^3$.

В целом, на~рабочем месте должны быть предусмотрены меры защиты 
от~возможного воздействия опасных и~вредных факторов производства. 
Уровни этих факторов не должны превышать предельных значений, 
оговоренных правовыми, техническими и~санитарно-техническими нормами. 
Нормативные документы обязывают к созданию на~рабочем месте условий
труда, при которых влияние опасных и~вредных факторов на~работающих либо
устранено совсем, либо находится в допустимых пределах.
Эти, а~также многие другие факторы необходимо учитывать при длительной 
и~интенсивной работе с~компьютером, такой как разработка дипломного проекта.
Соблюдение правил безопасности при работе с~компьютером позволяет не
только сохранить здоровье, но и~повысить производительность, уменьшив
утомление от длительного взаимодействия с~техникой.

