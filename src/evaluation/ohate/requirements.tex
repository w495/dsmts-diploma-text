%%%%%%%%%%%%%%%%%%%%%%%%%%%%%%%%%%%%%%%%%%%%%%%%%%%%%%%%%%%%%%%%%%%%%%%%%%%%%%%%
%%%
%%%  Основная часть
%%%

\subsection{Основная часть}

В основной части работы будут кратко описана большая часть требований 
к~условиям труда программиста. 
Аналогичные требования должны быть предъявлены
к~условиям труда переводчиков.
В конце части более подробно описан  уровень шума, 
возникающий от нескольких некогерентных источников.

\subsubsection{Микроклимат}

\index{микроклимат}
Микроклимат --- суть, состояние внутренней среды помещения, оказывающее воздействие 
на человека, характеризуемое показателями температуры воздуха 
и ограждающих конструкций, влажностью и подвижностью воздуха.

% В случае работы за компьютером таким помещением является офис.

Главным процессом, который регулируется параметрами микроклимата
является теплообмен человека с окружающей средой. 
Человеческому организму очень важно поддерживать постоянную температуру тела.
Это является необходимым условием жизнедеятельности человека, 
осуществляемым благодаря процессу терморегуляции. 

\index{терморегуляция}
Терморегуляция --- способность человека поддерживать температуру тела в определенных рамках, несмотря 
на~температуру окружающей среды. 

\index{гипертермия}
\index{гипотермия}
Отклонение нормальной для организма температуры в сторону увеличения называется гипертермией, 
в сторону уменьшения --- гипотермией. 
Главное, что может предоставить комфортный для человека микроклимат 
--- это оптимальные условия для теплообмена тела человека с окружающей средой.

При работе за компьютером в замкнутом помещении человек, как правило, 
окружен вычислительной техникой, основная особенность которой 
--- большое количество выделяемого в окружающую среду тепла. 

\index{влажность}
\index{подвижность!воздуха}
Это является следствием того, что в помещении происходит повышение температуры 
и снижение относительной влажности воздуха. 

\index{СН-245-71}
Именно температура и относительная влажность воздуха являются двумя основными параметрами, 
регулируемыми в~санитарных нормах \mbox {СН-245-71}.  
Подвижностью воздуха в пощение --- еще один важный параметр микроклимата.
Значения этих параметров должны зависеть от времени года. 

\begin{dtable}{Параметры микроклимата для помещений, где установлены компьютеры для холодного времени года.}
	\begin{tabular}{|r|l|}
		\hline  Параметр микроклимата  & Величина	 \\ 
		\hline  Температура воздуха в помещении  &  22.24°С \\ 
		\hline  Относительная влажность  &  40.60 \% \\ 
		\hline  Скорость движения воздуха &  до 0.1м/с \\ 
		\hline 
	\end{tabular}
\end{dtable}	

\begin{dtable}{Параметры микроклимата для помещений, где установлены компьютеры для теплого времени года.}
	\begin{tabular}{|r|l|}
		\hline  Параметр микроклимата  & Величина \\ 
		\hline  Температура воздуха в помещении  &  23.25°С \\ 
		\hline  Относительная влажность  &  40.60 \% \\ 
		\hline  Скорость движения воздуха &  0.1 - 0.2м/с \\ 
		\hline 
	\end{tabular} 
\end{dtable}
В помещение, где находятся компьютеры, 
необходимо осуществлять  подачу свежего воздуха.
Этот параметр зависит от объема помещения, 
приходящегося на одного человека, 
который не должен быть меньше, чем 19,5м$^3$ на человека.
\begin{dtable}{Расход подаваемого в~помещение свежего воздуха}
	\begin{tabular}{|r|p{7cm}|}
		\hline  Объем помещения, м$^3$ на чел.  & Объемный расход подаваемого в
			помещение свежего воздуха, м$^3$ на чел. в час	 \\ 
		\hline  До 20		&  Не менее 30  \\ 
		\hline  От 20 до 40 &  Не менее 20  \\ 
		\hline  От 40  		&  Естественная вентиляция  \\ 
		\hline 
	\end{tabular} 
\end{dtable}


Обычно, для достижения значений параметров, приведенных в таблицах 
выше применяют технические средства --- кондиционирование
воздуха, отопительная система и организационные методы --- рациональная
организация проведения работ в зависимости от времени года и суток,
чередование труда и отдыха.

\subsubsection{Режимы труда и отдыха}

Соблюдение правильного режима может значительно
улучшить самочувствие и повысить производительность труда.

В случае несоблюдения режимов труда и отдыха у человека 
при долгой работе за компьютером могут наблюдаться:

\begin{itemize}
	\item	усталость;
	\item	нарушения сна;
	\item	болезненные ощущения 
		(в глазах, пояснице и области шеи).
\end{itemize}

\index{СанПиН 2.2.2 546-96}
В соответствии со \mbox{СанПиН 2.2.2 546-96} все виды трудовой деятельности, 
связанные с использованием компьютера, разделяются на три группы:

\begin{itemize}
	\item  группа А: работа по считыванию информации 
		с экрана компьютера с предварительным запросом;
	\item  группа Б: работа по вводу информации;
	\item  группа В: творческая работа в режиме диалога с ЭВМ. 
\end{itemize}

\begin{dtable}{Группа А}
	\begin{tabular}{|m{4cm}|r|r|}
		\hline	Уровень нагрузки за~рабочую смену, количест­во знаков
			& \multicolumn{2}{m{7cm}|}{Суммарное время регламентированных перерывов, минут } \\
		\cline{2-3}   															&	Смена 8 часов 	&	Смена 12 часов  \\
		\hline	до 20000
																					& 	30 		  &		70		    \\
		\hline	до 40000
																					& 	50 		  &		90		    \\
		\hline	до 60000 
																					& 	70 		  &		120		    \\
		\hline 
	\end{tabular} 
\end{dtable}

\begin{dtable}{Группа Б}
	\begin{tabular}{|m{4cm}|r|r|}
		\hline	Уровень нагрузки за~рабочую смену, количест­во знаков
			& \multicolumn{2}{m{7cm}|}{Суммарное время регламентированных перерывов, минут } \\
		\cline{2-3}   															&	Смена 8 часов 	&	Смена 12 часов  \\
		\hline	до 15000
																					& 	30 		  &		70		    \\
		\hline	до 30000
																					& 	50 		  &		90		    \\
		\hline	до 40000 
																					& 	70 		  &		120		    \\
		\hline 
	\end{tabular} 
\end{dtable}


\begin{dtable}{Группа В}
	\vspace{1cm}
	\begin{tabular}{|m{4cm}|r|r|}
		\hline	Уровень нагрузки за~рабочую смену, часов
			& \multicolumn{2}{m{7cm}|}{Суммарное время регламентированных перерывов, минут } \\
		\cline{2-3}   															&	Смена 8 часов 	&	Смена 12 часов  \\
		\hline	до 2 
																					& 	30 		  &		70		    \\
		\hline	до 4
																					& 	50 		  &		90		    \\
		\hline	до 6
																					& 	70 		  &		120		    \\
		\hline 
	\end{tabular} 
	\vspace{1cm}
\end{dtable}

Время перерывов дано при соблюдении указанных Санитарных правил и норм. 
При несоответствии фактических условий труда требованиям Санитарных правил 
и норм время регламентированных перерывов следует увеличить на 30 \%. 

Эффективность перерывов повышается при сочетании с производственной 
гимнастикой или организации специального помещения для отдыха персонала 
с удобной мягкой мебелью, аквариумом, зеленой зоной и т.п. 


\subsubsection{Электромагнитное и ионизирующее излучения}

По-мнению ученых, излучение большинства современных мониторов 
не~оказывает пагубного воздействия для взрослого человека. 

Тем не менее, исчерпывающих данных по этому вопрос пока нет.

Максимальный уровень рентгеновского излучения от монитора составляет в среднем 10 $\dfrac{\text{мкБэр}}{\text{ч}}$, 
а интенсивность ультрафиолетового и инфракрасного излучений лежит в интервале 10-100 $\dfrac{\text{мВт}}{\text{м}^2}$.

\index{СанПиН 2.2.2 542-96}

Ниже описаны допустимые значения параметров
неионизирующих электромагнитных излучений 
(в соответствии с \mbox{СанПиН 2.2.2.542-96}).

\begin{itemize}
	\item  
		Напряженность электрической составляющей
		электромагнитного поля на расстоянии $50$ см от
		поверхности видеомонитора --- $10 \dfrac{\text{В}}{\text{м}}$.
	\item  
		Напряженность магнитной составляющей
		электромагнитного поля на расстоянии $50$ см от
		поверхности видеомонитора --- $0.3 \dfrac{\text{А}}{\text{м}}$.
	\item  
		Для взрослых пользователей напряженность 
		электростатического поля не должна превышать --- $20 \dfrac{\text{кВ}}{\text{м}}$.
\end{itemize}

\index{MPR-II}
\index{TCO-92}
\index{TCO-99}

Для снижения воздействия этих видов излучения рекомендуется применять
мониторы с пониженным уровнем излучения (MPR-II, TCO-92, TCO-99),
устанавливать защитные экраны, а также соблюдать регламентированные
режимы труда и отдыха.

\subsubsection{Освещение}

Обычно освещению как рабочего места, так и помещения уделяют мало внимания. 
При работе за компьютером несоблюдение правил освещения является
одной из основных причин ухудшения зрения. 

Недостаточность освещения:
\begin{itemize}
	\item  ослабляет внимание;
	\item  приводит к наступлению преждевременной утомленности.
\end{itemize}

Чрезмерно яркое освещение:
\begin{itemize}
	\item  вызывает ослепление;
	\item  раздражение и резь в глазах;
	\item  приводит к наступлению преждевременной утомленности.
\end{itemize}

Неправильное направление света на рабочем месте может создавать резкие
тени, блики. Для программиста, эти факторы не являются особенно важными, 
однако, благоприятными их тоже назвать нельзя.

Существует три вида освещения --- естественное, искусственное и совмещенное
(естественное и искусственное вместе).

Естественное освещение --- освещение помещений дневным светом, проникающим
через световые проемы в наружных ограждающих конструкциях помещений.
Естественное освещение характеризуется тем, что меняется в широких пределах
в зависимости от времени  дня, времени года, характера области и ряда других
факторов.

Искусственное освещение применяется при работе в темное время суток и днем,
когда не удается обеспечить нормированные значения коэффициента
естественного освещения (пасмурная погода, короткий световой день).

Искусственное освещение бывает трех видов: 

\begin{itemize}
	\item  рабочее;
	\item  аварийное;
	\item  эвакуационное.
\end{itemize}

Освещение, при котором недостаточное по нормам естественное освещение
дополняется искусственным, называется совмещенным освещением.

Рабочее освещение, может быть общим или комбинированным.

Общее – освещение, при котором светильники размещаются в верхней зоне
помещения равномерно или применительно к расположению
оборудования. 

Комбинированное – освещение, при котором к общему 
добавляется местное освещение.

\index{СНиП II-4-79}

Согласно \mbox{СНиП II-4-79} в помещений вычислительных центров необходимо
применить систему комбинированного освещения.

В качестве источников искусственного освещения обычно
используются люминесцентные лампы типа ЛБ или ДРЛ. Они попарно
объединяются в светильники, которые должны располагаться над рабочими
поверхностями равномерно.

В физике освещенность определяется как отношение светового потока на
единицу поверхности:
\[
	E = \dfrac{d \Phi}{d S}
\]

Единицей измерения освещенности в системе CИ служит люкс (лк).

Световой поток можно рассчитать по формуле:
\[
	\Phi = \dfrac{E \cdot K \cdot S \cdot Z}{n}
\]

\begin{itemize}
	\item  $\Phi$ - рассчитываемый световой поток, лм;
	\item  $E$ - нормированная минимальная освещенность, лк (определяется по
		таблице). Работу программиста, в соответствии с этой таблицей, можно отнести
		к разряду точных работ, следовательно, минимальная освещенность будет $Е =
		300$~лк (значения для минимальной освещенности описаны ниже);
	\item  $S$ - площадь освещаемого помещения;
	\item  $Z$ - отношение средней освещенности к минимальной (обычно принимается равным $1.0$, $1.1$, $2.0$);
	\item  $K$ - коэффициент запаса, учитывающий уменьшение светового потока
		лампы в результате загрязнения светильников в процессе эксплуатации 
		(его значение зависит от типа помещения и характера проводимых в нем работ);
	\item  $n$ - коэффициент использования, (выражается отношением светового
		потока, падающего на расчетную поверхность, к суммарному потоку всех ламп
		и исчисляется в долях единицы; зависит от характеристик светильника,
		размеров помещения, окраски стен и потолка, характеризуемых
		коэффициентами отражения от стен и потолка).
\end{itemize}

Требования к освещенности в помещениях, где установлены компьютеры,
следующие: при выполнении зрительных работ высокой точности общая
освещенность должна составлять $300$~лк, а комбинированная --- $750$~лк;
аналогичные требования при выполнении работ средней точности --- $200$~и~$300$~лк
соответственно.
Кроме того все поле зрения должно быть освещено достаточно равномерно ---
это основное гигиеническое требование.

\subsubsection{Шум и вибрация}

При работе с любой техникой необходимо соблюдать спокойствие, 
не давать волю эмоциям. В противном случае, как для человека, 
так и для технического средства могут наступить необратимые последствия.
Одним из наиболее сильных раздражающих факторов является шум.

Под воздействием шума ухудшается внимание, повышается раздражительность. 
Последнее является весьма опасным последствием работы за компьютером.
Человек испытывает головные боли, головокружение. 

Шум замедляет реакцию человека на поступающие от технических устройств сигналы. 
Шум угнетает центральную нервную систему, вызывает 
изменения скорости дыхания и~пульса, способствует нарушению обмена веществ,  \\
возникновению сердечно-сосудистых заболеваний. 

В случае, когда шум воздействует на слуховые органы человека постоянно 
и~его эффективность велика, то это может привести \\
к~частичной или полной потере слуха человека.

Одна из измеримых характеристик звука --- это количество заключенной в нем
энергии; интенсивность звука в любой точке можно измерить как поток
энергии, приходящейся на единичную площадку, и выразить в
ваттах на квадратный метр $\left( \dfrac{\text{Вт}}{\text{м}^2} \right) $. 

Такая характеристика не удобна. Возможен очень широком диапазон значений. 

При попытке записать в этих единицах интенсивность обычных шумов сразу
же возникают трудности, так как интенсивность наиболее тихого звука,
доступного восприятию человека с самым острым слухом, \\
равна приблизительно $ 0.1 \cdot 10^{-11} \quad \dfrac{\text{Вт}}{\text{м}^2}$. 

Легко видеть, что оперировать числами, выражающими интенсивности звука,
лежащие в столь широком диапазоне, очень трудно. Выходом из сложившейся
ситуации является использование некоторой \textit{относительной величины}. 

\index{децибел}
Такой величиной является децибел [дБ]. Уровень звука, выраженный в децибелах,
численно равен десятичному логарифму безразмерного отношения измеряемой
интенсивности звука к эталонной интенсивности звука, умноженному на
десять.

\[
	A_{\text{дБ}} = \log_{10} \left( \dfrac{A_1}{A_0} \right)
\]

\begin{itemize}
	\item  $A_1$ – измеряемая интенсивность;
	\item  $A_0$ – эталонная интенсивность, принимаемую за $ 10-12  \quad \dfrac{\text{Вт}}{\text{м}^2}$.
\end{itemize}

Напомним, что децибел --- это относительная величина. 
Операции над~ней отличаются от операций над~абсолютными величинами. 
Общий уровень шума от двух источников, выраженный в децибелах, 
не будет равен сумме каждого из них. 
Суммировать необходимо интенсивность двух источников, 
а~после этого перейти к децибелам путем увеличения логарифма.

Уровень шума, возникающий от нескольких некогерентных источников, работающих
одновременно, подсчитывается на основании принципа энергетического
суммирования излучений отдельных источников

\index{уровень звукового давления}
\[
	L_{\varSigma} = 10 \cdot log_{10} \left( \sum\limits_{i = 1}^{i = n}   \left(    10^{0.1 \cdot L_i}    \right)  \right)
\]

\begin{itemize}
	\item  $L_i$ – уровень звукового давления $i$-го источника шума;
	\item  $n$ – количество источников шума.
\end{itemize}

Полученные результаты расчета сравнивается с допустимым значением уровня
шума для данного рабочего места. Если результаты расчета выше допустимого
значения уровня шума, то необходимы специальные меры по снижению шума.
Для того, чтобы оценить уровень шума в помещении, обычно 
используются специализированные устройства --- шумомеры.
В простейшем случае шумомер состоит из усилителя, 
к входу которого подключается измерительный микрофон, 
а к выходу – вольтметр, проградуированный в децибелах.
Однако, существуют иные способы измерения. 
Для этого можно воспользоваться обычным микрофоном 
и любым профессиональным звуковым редактором \textit{(Sound Forge, Nero Wave Editor)}. 
Микрофон и программное обеспечение 
предварительно следует откалибровать.
Для оценки шума такого сложного устройства как компьютер может потребоваться,
провести эксперименты по измерению уровня шума для каждой составляющей в отдельности.
В противном случае придется поверить шумовым характеристикам, которые указанны производителями комплектующих.

\begin{dtable}{Характеристики шума типичного настольного компьютера.}
	\begin{tabular}{|r|r|}
		\hline  Источник шума   		& Уровень шума, дБ	 \\ 
		\hline  Жесткий диск 		 	&  $35$ \\ 
		\hline  Система охлаждения		&  $45$ \\ 
		\hline  Монитор	 				&  $17$ \\ 
		\hline  Клавиатура  		  	&  $5$  \\ 
		\hline  Принтер	 				&  $40$ \\ 
		\hline 
	\end{tabular} 
\end{dtable}

\pagebreak

Таким образом можно оценить шум типичного настольного компьютера.
\[
	L_{\varSigma} = 10 \cdot log_{10} \left( 
		10^{3.5} +  10^{4.5} + 		10^{1.7} + 		10^{0.5} + 		10^{4.0}	
	 \right) =
\]\[
	 = 10 \cdot log_{10} (44838.3) = 46.5165 \text{ дБ}
\]

\begin{dtable}{Уровни звука в децибелах на различных рабочих местах.}
	\begin{tabular}{|m{5cm}|c|c|c|c|}
		\hline	Категория напряженности труда	  & \multicolumn{4}{c|}{Категория тяжести труда} \\ 
		\cline{2-5}
				&	Легкая	&	Средняя 	&	Тяжелая		&	Очень тяжелая \\
		\hline	Мало напряженный
				& 	$80$	&		$80$	&		$75$	&		$75$	\\
		\hline	Напряженный
				& 	$70$	&		$70$	&		$65$	&		$65$	\\
		\hline	Очень напряженный
				& 	$60$	&		$60$	&		---		&		---		\\
		\hline	Опасно напряженный	
				& 	$50$	&		$50$	&		---		&		---		 \\
		\hline 
	\end{tabular} 
\end{dtable}

Уровень шума на рабочем месте программистов не должен превышать $50$ дБ, 
а~в залах обработки информации на вычислительных машинах --- $65$ дБ. 
Вычисленное значение не превышает допустимый уровень шума рабочего места. 
Однако, важно понимать, что приведенные расчеты не претендуют на высокую точность.
Реальный уровень может быть несколько больше, так как вычислительная 
машина является далеко  не единственным источником шума.
Для~обеспечения показателей установленных нормой, необходимо использовать
звукопоглощающие материалы и виброизоляторы, в которые помещается
оборудование. Во многих случаях рекомендуется размещать 
рабочее место программиста и само устройство удаленно друг~от~друга.

\pagebreak
