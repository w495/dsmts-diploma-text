%%%%%%%%%%%%%%%%%%%%%%%%%%%%%%%%%%%%%%%%%%%%%%%%%%%%%%%%%%%%%%%%%%%%%%%%%%%%%%%%
%%%
%%% Расчет затрат на разработку
%%%

\subsection{Расчет затрат на разработку}

Всю работу над проектом можно разбить на следующие этапы
\begin{enumerate}
	\item  Анализ проблемы, выделение ключевых задач и действий 
		\mbox{(работа 0 --- 1)}.
	\item  Исследование принципов работы существующих систем статистического перевода, 
		изучение теоретических основ \\ \mbox{(работы 1 --- 2, 1 --- 3, 1 --- 4, 1 --- 5, 2 --- 6 )}.
	\item  Разработка  численных алгоритмов для работы системы \\ \mbox{(работы 3 --- 7, 3 --- 8, 4 --- 9, 4 --- 8)}.
	\item  Разработка структуры хранения данных и распеределенной архитектуры \\ \mbox{(работы 5 --- 7, 6 --- 10)}.
	\item  Реализация отдельных модулей приложения \\ \mbox{(работы 7 ---10, 8 --- 13, 10 --- 12)}.
	\item  Прогон системы на данных приближенных к реальности, корректировка системы \mbox{(работы 9 ---11, 11 --- 12, 12 --- 13)}.
	\item  Тестирование и отладка \mbox{(работы 13 --- 14)}.
	\item  Прогон системы на реальных данных \mbox{(работы 14 --- 15)}.
\end{enumerate}

Одной из основных статей расходов является заработная плата персонала, 
занятого в исследованиях и разработке при проведении данной дипломной работы. 
Расчет среднемесячной и среднедневной зарплаты работников,\\
задействованных в проекте, приведен ниже.

\pagebreak
\begin{dsmalltable}{ Расчет среднемесячной и среднедневной зарплаты работников.}
	\begin{tabular}{|r|p{3cm}|p{3cm}|p{3cm}|p{2cm}|p{2.5cm}|}
		\hline
			№ & ИТР 
				& Количество сотрудников 
				& Среднемесячная зарплата (руб.) 
				& Количество рабочих дней в~месяце 
				&  Среднедневная зарплата (руб.) \\
		\hline
			1 & Системный архитектор & $1$ & $60000$ & $21$ &  $2857.14$ \\
		\hline
			2 & Лингвист 			 & $1$ & $60000$ & $21$ &  $2857.14$ \\
		\hline
			3 & Математик 			 & $2$ & $60000$ & $21$ &  $2857.14$ \\
		\hline
			4 & Разработчик 		 & $2$ & $60000$ & $21$ &  $2857.14$ \\
		\hline
			5 & Тестировщик 		 & $1$ & $60000$ & $21$ &  $2857.14$ \\
		\hline
	\end{tabular}
\end{dsmalltable}

Трудоемкость вычисляется по формуле:
\[
	Q(i - j) = t(i - j) \cdot A(i - j) \cdot f
\]
\begin{itemize}
	\item  $i$, $j$ --- начальное и конечное события работы $E(i - j)$;
	\item  $Q(i - j)$ --- трудоемкость работы, чел/дн.;
	\item  $A(i - j)$ --- количество исполнителей, занятых выполнением работы $E(i - j)$;
	\item $f$ --- коэффициент перевода (при необходимости) рабочих дней в календарные, 
		$f = 0.85$ для пятидневной рабочей недели или $f = 1.0$, если перевод в календарные дни не требуется. 
\end{itemize}
В данном случае коэффициент перевода берется $0,85$.
Вычислим расходы на зарплату персонала этапам работ обозначенным выше.
Важно заметить, что для простоты в таблице зарплаты различных работников 
не различаются. В таблице выше мы уже привели зарплаты сотрудников --- они одинаковы.
В противном случае, придется вычислить среднедневную заработную плату, приходящуюся на одного
человека в команде (не зависимо от его роли), и далее оперировать этой цифрой.

\pagebreak

\begin{dsmalltable}{Расходы на зарплату персонала.}
	\begin{tabular}{|m{1cm}|m{3cm}|m{3cm}|m{3cm}|m{3cm}|}
		\hline
			№ этапа 
				& Количество исполнителей (чел.)
				& Трудоёмкость (чел./дн.) 
				& Среднедневная зарплата (руб.)
				& Зарплата (руб.) \\
		\hline
			1 & 1 &  $3.23$ & $2857.14$ &  $9228.56$ \\
		\hline
			2 & 4 & $28.55$ & $2857.14$ &  $81599.91$ \\
		\hline
			3 & 4 & $13.42$ & $2857.14$ &  $38371.39$ \\
		\hline
			4 & 2 &  $4.93$ & $2857.14$ &  $14085.70$ \\
		\hline
			5 & 3 & $13.43$ & $2857.14$ &  $38371.39$ \\
		\hline
			6 & 3 &  $7.31$ & $2857.14$ &  $20885.69$ \\
		\hline
			7 & 6 & $65.28$ & $2857.14$ &  $186514.09$ \\
		\hline
			8 & 1 & $1.275$ & $2857.14$ &  $3642.853$ \\
		\hline
	Итого &       & $137.425$ &		    &  $392642.46$  \\
		\hline
	\end{tabular} 
\end{dsmalltable}

Из расчетов выше, следует, что суммарная трудоемкость 
($\approx 137$ человеко-дней) значительно превышает длину критического пути ($47$ дней). \\
Это свидетельствует о том, что при данном планировании персонал используется достаточно эффективно.
Основным результатом расчета этой таблицы является выявление суммы затрат на заработную плату.
\[
	S_{\text{зп}} = 392642.46 \text{ рублей}.
\]
Затрат на закупку программного обеспечения нет, 
т.к. для разработки
планируется использовать открытые продукты.

\begin{dtable}{Затраты на оборудование.}
	\begin{tabular}{|r|p{4cm}|r|r|r|}
		\hline
				№ 
				& Наименование
				& Количество
				& Цена (руб.)
				& Стоимость (руб.) \\
		\hline
			1 & Ноутбук Sony Vaio &  $7$ & $40000.00$   &    $280000.00$ \\
		\hline
	Итого &     							& 	  &  					  &   $280000.00$  \\
		\hline
	\end{tabular} 
\end{dtable}

Суммарная стоимость оборудования:
\[
	S_{\text{об}} = 280000.00 \text { рублей}.
\]

Рассмотрим оплату Интернета как дополнительную статью расходов:
\[
	S_{\text{Интернета}} = S_{\text{подключения}}  +  t \cdot S_{\text{мес}}
\]
\begin{itemize}
	\item  $S_{\text{подключения}}  = 1500 $ стоимость подключения, руб;
	\item  $S_{\text{мес}} = 600 $ --- оплата безлимитного тарифа в месяц, руб;
	\item  $t$ --- количество месяцев разработки, из расчета, что в месяце только 21 рабочий день;
\end{itemize}
\[
	S_{\text{Интернета}} = 1500 +  3 \cdot 600 = 3300.00 \text{ рублей}.
\]

Кроме того, для повышения качаства системы, возможно, ее придется тестировать на платных корпусах текста.
Например:
\begin{itemize}
	\item  <<European Corpus Initiative Multilingual Corpus I>>;
	\item  <<Национальный корпус русского языка>>.
\end{itemize}

<<European Corpus Initiative Multilingual Corpus I>> доступен по цене 2000 рублей, в бессрочное пользование любого характера.
Про коммерческую доступность <<Национального корпуса русского языка>> ничего пока не известно, потому мы не будем его учитывать.
Тогда
\[
	S_{\text{корп}} = 2000  \text{ рублей}.
\]
Суммарные расходы на разработку могут быть вычислены по формуле:
\[
	S = 	S_{\text{зп}} \cdot (1 + \omega_{d}) \cdot  (1 + \omega_{c}) + S_{\text{об}} + S_{\text{Интернета}} + S_{\text{корп}}
\]
\begin{itemize}
	\item  $\omega_{d} = 0.2$ --- коэффициент, учитывающий дополнительную заработную плату (премии);
	\item  $\omega_{c} = 0.34$ --- коэффициент, учитывающий страховые взносы \\ во внебюджетные фонды.
\end{itemize}
\[
	S = 916669.08 \text{рублей}.
\]
Далее, вычислим цену полученной системы:
\[
	C = \dfrac{(1 + P_{\text{н}}) \cdot S}{n};
\]
\begin{itemize}
	\item  $P_{\text{н}} = 0.2$ --- норматив рентабельности, учитывающий часть чистого дохода, 
		включенного в цену (может быть принят равным $0,2$);
	\item  $n$ --- количество организаций, которые могут купить разрабатываемое программное обеспечение.
\end{itemize}
Подобная система может оказаться полезной прежде всего крупным бюро переводов, 
и крупным многоязычным интернет-порталам. Оценить число последних не представляется возможным.
Общее число бюро переводов зарегистрированных в г. Москве насчитывает примерно  $\approx 600$.
Будем считать, что, потенциально,  каждое из них может купить данную систему.
Тогда,
\[
	C = 1833.33 \text{ рублей}.
\]
Вообще, подобная система может оказаться полезной:
\begin{itemize}
	\item издательствам, занимающимся переводом иностранной технической литературы;
	\item ведомственным организациям;
	\item военным организациям;
	\item конструкторским бюро и исследовательским центрам.
\end{itemize}
Однако, эти типы предприятий в данной оценке, мы использовать не будем.
Кроме того, авторы работы убеждены, что системы подобного 
класса должны поставлять государственным учреждениям бесплатно.
Капитальные вложения, связанные с внедрением 
в организации-пользователе новой программы, 
равны продажной стоимости системы. 
На данный момент эта стоимость составляет $1833.33$  рублей. 
Расходы, связанные с эксплуатацией системы (на одну единицу техники, в год) могут быть определены следующим образом:
\[
	U = T_{\text{м.в.}} \cdot C_{\text{м.в.}}  + \dfrac{C}{T_{с}} 
\]
\begin{itemize}
	\item  $T_{с} = 0.5 $ --- срок морального устаревания системы (условно примем за~полгода) ;
	\item  $T_{\text{м.в.}}$  --- годовое машинное время вычислительной машины, 
		необходимое для применения внедряемой системы 
			(в~данном случае наиболее эффективно использовать вычислительный кластер, 
			но с~учетом высокой цены такого оборудования, 
			вполне может подойти выделенный сервер, 
			или даже простой персональный компьютер);
	\item $C_{\text{м.в.}}  = 12 $ рублей, стоимость часа машинного времени.
\end{itemize}

В~данном случае, мы полагаем, что каждое из рассматриваемых бюро переводов обладает своей базой текстов.
Таким образом, не будет необходимости
в~дополнительных расходах, на покупку сторонних платных корпусов текстов.
В итоге получаем:
\begin{itemize}
	\item  при работе систему $760$ часов в году:
		$
			U_{\text{маш}, 760} = 12786.00 \quad \text{ рублей}
		$;
	\item  при работе систему $1993$ часов в году ($250$ рабочих дней, при $40$-часовой рабочей неделе):
		$
			U_{\text{маш}, 1993} = 27582.67 \quad \text{ рублей}
		$;
	\item  при работе систему $8760$ часов в году (в режиме $24$ на $7$ на $365$):
		$
			U_{\text{маш}, 8760} = 108786.67 \quad \text{ рублей}
		$.
\end{itemize}

Кроме того, для использования системы, предприятию придется расширить парк машин.
Однако, эти капиталовложения могут зависеть от нужд и возможностей самой компании,
и колеблются в диапазоне от $12000$ рублей до $13$ млн. рублей 
(покупка вычислительного кластера класса Блэйд с~$40$ узлами). 

