%%%%%%%%%%%%%%%%%%%%%%%%%%%%%%%%%%%%%%%%%%%%%%%%%%%%%%%%%%%%%%%%%%%%%%%%%%%%%%%%
%%%
%%% Заключение
%%%

\subsection{Целесообразность применения системы}

Системы машинного перевода целесообразно применять только для перевода 
научно-технической литературы. 
Это обусловлено стилистическими особенностями научного текста. 
В других случаях результат машинного перевода
не представляет какой-либо ценности.

Системы машинного перевода (пока) не могут полностью заменить человека,
однако, они значительно облегчают труд переводчика, делая его работу более 
эффективной. 

\pagebreak

Кроме того, системы машинного перевода могут быть необходимы:
\begin{itemize}
	\item  научным сотрудникам исследовательских центров --- чтобы в кратчайшие сроки получить 
			общее представление, о той или иной работе, \\ на иностранном языке;
	\item  военным и ведомственным организациям --- когда уровень секретности, 
		не всегда позволяет переводчику нужной специализации получить доступ к исходному тексту.
\end{itemize}
Основное преимущество данной системы, заключается, в том, 
что ее можно распространять в <<коробочном>> варианте
(распределенность желательна, но совсем не обязательна).
Это является гарантией, что информация введенная для перевода 
не будет доступна сторонним лицам.
Кроме того, благодаря, тому что система основана на статистике, 
ее можно настроить на тексты, заданной тематики,
что будет весьма полезно первой и второй категории пользователей.

Системы машинного перевода могут быть выгодны:
\begin{itemize}
	\item  крупным бюро перевода;
	\item  издательствам, занимающимся переводом иностранной технической литературы;
	\item  инернет-порталам.
\end{itemize}

Заработная плата переводчика на конец $2011$ года колеблется от $20000$ руб до $45000$ руб.
Если взять среднюю величину и выяснить во сколько организации обходится один переводчик в год,
то получится
\[
	U_{\text{чел}, avg} = 
		\dfrac{45000 + 20000}{2}  \cdot 12  \cdot (1 + \omega_{d}) 
		\cdot  (1 + \omega_{c}) = 627120.00  \text{ рублей}.
\]\[
	U_{\text{чел}, min} = 
		20000  \cdot 12 \cdot (1 + \omega_{d}) 
		\cdot  (1 + \omega_{c}) = 385920.00  \text{ рублей}.
\]

\pagebreak
Напомним:
\begin{itemize}
	\item  $\omega_{d} = 0.2$ --- коэффициент, учитывающий дополнительную заработную плату (премии);
	\item  $\omega_{c} = 0.34$ --- коэффициент, учитывающий страховые взносы \\ во внебюджетные фонды.
\end{itemize}
Основываясь на расчетах проведенных в предыдущем разделе, совокупная стоймость владения
рассматриваемой системы в самом дорогом случае (без учета капиталовложений в оборудование) ниже,
чем  компании обходится самый низкооплачиваемый переводчик.
\[
	U_{\text{маш}, 8760} = 108786.67 < 385920.00 = U_{\text{чел}, min}
\]
Причем, если рассматривать, только рабочие дни и 40-часовую рабочую неделю, то разница становится более ощутимой.
\[
	U_{\text{маш}, 1993} = 27582.67 < 385920.00 = U_{\text{чел}, min}
\]
Кроме того, скорость профессионального переводчика обычно ограничена двадцатью страницами в день,
в то время как, на тот же объем текста, машина потратит в худшем случае несколько минут.

Важно понимать, что статистические системы машинного 
перевода не могут полностью функционировать без человека.
Перед тем как начать переводить, статистические СМП должны обучиться на переводах, которые сделал человек.
Причем от качества текстов, на которых машина обучается, будет зависеть и качество результата ее работы.
Перевод выдаваемый машиной не всегда удовлетворяет литературным стандартам языка перевода.
Потому в этих случаях могут потребоваться услуги корректора.
Однако, стоит заметить, что при работе переводчика-человека также бывает необходим корректор.
Чаще в его роли выступает переводчик более высокой квалификации.
В итоге, можно придти к следующей формуле
\[
	U_{\text{чел}, max} + U_{\text{маш}, 8760} < U_{\text{чел}, min}  + U_{\text{чел}, max} 
\]
где, $U_{\text{чел}, max}$ --- расходы на заработную плату корректора или переводчика высокой квалификации.
(В данном случае переводчиков с низкой квалификацией, можно отправить повышать свою квалификацию.)

На предприятиях, основной деятельностью, которых является перевод текстов 
с одного языка на другой возможно следующей применение схемы 
\[
	\text{\sffamily человек}_1 \rightarrow \text{\sffamily машина}  \rightarrow \text{\sffamily человек}_2
\]
\begin{itemize}
	\item  $\text{\sffamily человек}_1$ --- высококвалифицированный переводчик, 
		который продолжает переводить особо важные тексты, самостоятельно, без применения СМП;
	\item  $\text{\sffamily машина}$ --- статистическая СМП, которая предварительно обучается. на переводах $\text{\sffamily человека}_1$;
	\item  $\text{\sffamily человек}_2$ --- высококвалифицированный переводчик, который корректирует результаты СМП.
\end{itemize}

Выгодность данной схемы требует дополнительного обоснования 
и практических измерений скорости работы людей и СМП
и выходит за рамки данной работы.
	
\pagebreak
