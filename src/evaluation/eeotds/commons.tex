%%%%%%%%%%%%%%%%%%%%%%%%%%%%%%%%%%%%%%%%%%%%%%%%%%%%%%%%%%%%%%%%%%%%%%%%%%%%%%%%
%%%
%%% Общие положения
%%%

\subsection{Введение}

По мере того как расширяется информатизация современного общества, 
возрастает значение прикладной (вычислительной, компьютерной, инженерной) 
лингвистики, науки, находящейся на стыке глубоко человечной, гуманитарной 
науки лингвистики (языкознания), изучающей законы развития 
и пользования могучим средством мышления и коммуникации --- языком, --- 
и компьютерного знания, с помощью которого машине передастся 
все большая часть интеллектуального труда человека \cite{Марчук:2000}.

В данной главе будет рассмотрена экономическая сторона вопроса. 
Вначале будет построена сетевая модель работ над проектом и её графическое представление. 
Будут рассчитаны ранние и поздние сроки начала и завершения работ над проектом, 
найден критический путь и его продолжительность, вероятность завершения 
комплекса работ по проекту в срок. После этого будет рассчитана сумма 
расходов на разработку проекта. Будет подсчитана цена разработанной системы, 
капитальные вложения, связанные с её внедрением, а также расходы, 
связанные с её эксплуатацией.

