\subsubsection{Сравнение с решениями других производителей}

На данный момент существует несколько открытых платформ 
для создания систем машинного перевода.
Наиболее крупные из них Moses и Chaski.
Принципы работы этих систем построенных на основе той и другой платформы 
во многом сходны, только с тем отличием, что Chaski позволяет строить распределенные системы. 
Для обучения в обоих случаях используется библиотека GIZA++, однако Chaski включает 
в себя многопроцессорную версию этой библиотеки.

Для тестирования разрабатываемой системы, были созданы еще две ССМП 
на основе Moses и Chaski.
Ниже приведены сравнения этих систем с разработанной.

\begin{dtable}{Сравнение модулей обучения для различных систем.}
	\begin{tabular}{|r|c|c|}
		\hline  \textbf{Система}		& \textbf{Распределенная} 	& \textbf{Отказоуст.} \\  
		\hline  Текущая 				& Да			 			& 	Да \\ 
		\hline  Moses (GIZA++)			& Нет			 			& 	Нет \\ 
		\hline  Chaski (MGIZA++)		& Да			 			& 	Нет \\ 
		\hline 
	\end{tabular} 
\end{dtable}
\begin{dtable}{Сравнение модулей декодирования для различных систем.}
	\begin{tabular}{|r|c|c|c|c|}
		\hline  \textbf{Система}		& \textbf{Распределенная} 	
			& \textbf{Отказоуст.}  & \textbf{Web} & \textbf{Rest}\\  
		\hline  Текущая 				& Да
			& 	{\centering Да  } & Да & Да  \\ 
		\hline  Moses (GIZA++)			& Возможно
			& 	{\centering Нет } & Да & Нет \\ 
		\hline  Chaski (MGIZA++)		& Возможно
			& 	{\centering Нет } & Возможно & Нет \\ 
		\hline 
	\end{tabular} 
\end{dtable}

Самым большим достоинством разработанной системы является то, 
что она ориентирована на перевод текстов исключительно научно-технического характера,
которые имеют явную стилистически подчеркнутую принадлежность к жанру научной прозы.
Такая узкая специализация системы является и основным недостатком.

\pagebreak

