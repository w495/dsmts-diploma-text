
Пусть $\FR$ --- фраза оригинала, русская. 
Требуется найти $\FE$ --- фразу перевода, английскую.
Нужно максимизировать $P(\FE|\FR)$. 
Если вспомнить модель зашумленного канала (модель Шеннона), то получаем:
\[
	P(\FE|\FR) =  \dfrac{\left( P(\FE) \cdot P(\FR|\FE) \right) }{P(\FR)}  \Rightarrow
\]
\[
	\FE_g = \arg\max\limits_{\cup \FE} P(\FE|\FR) =  \arg\max\limits_{\cup \FE} \left(  P(\FE) \cdot P(\FR|\FE) \right) 
\]
$P(\FR)$ --- нам известна, ее не учитываем.
Величина $ P(\FE) $ называется моделью языка. $ P(\FR|\FE) $ --- модель перевода.
Работа любой статистической системы перевода состоит из двух этапов:
\begin{itemize}
	\item обучения --- вычисляются модели языка и перевода;
	\item эксплуатации --- вычисляется величина $\arg\max\limits_{\cup \FE} P(\FE|\FR)$ при данной $\FR$
		(процесс вычисления называют декодированием).
\end{itemize}
 
