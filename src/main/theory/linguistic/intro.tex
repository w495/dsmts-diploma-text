%%%%%%%%%%%%%%%%%%%%%%%%%%%%%%%%%%%%%%%%%%%%%%%%%%%%%%%%%%%%%%%%%%%%%%%%%%%%%%%%
%%%
%%% LNG INTRO
%%%
	
	В настоящее время имеется достаточно широкий выбор пакетов программ, 
	облегчающих труд переводчика, которые условно можно подразделить на~две основные группы:
	\begin{itemize}
		\item  электронные словари;
		\item  системы машинного перевода.
	\end{itemize}
	
	Системы машинного перевода текстов с одних естественных языков 
	на~другие моделируют работу человека-переводчика. 
	Их~полезность зависит от того, в~какой степени 
	в~них учитываются объективные законы языка и мышления. 
	Законы эти пока еще изучены плохо. 
	Поэтому, решая задачу машинного перевода, 
	необходимо учитывать опыт межнационального общения 
	и~опыт переводческой деятельности, накопленный человечеством. 
	В~процессе перевода в~качестве основных единиц смысла выступают 
	не~отдельные слова, а~фразеологические словосочетания, выражающие понятия. 
	Именно~понятия являются элементарными мыслительными образами. 
	Только~используя их можно строить более сложные образы, 
	соответствующие переводимому тексту. 
	В~современной лингвистике можно выделить 
	ряд направлений использования компьютера:
	\begin{itemize}
		\item машинный перевод;
		\item отдельные виды автоматизации лингвистических исследований;
		\item автоматизация лексикографических работ;
		\item автоматический поиск библиографической информации.
	\end{itemize}
	
	В этой работе мы будем подробно рассматривать системы машинного перевода.
	На данный момент выделяют три типа систем машинного перевода:
	\begin{itemize}
		\item полностью автоматический;
		\item автоматизированный машинный перевод 
			при участии человека (MT\footnote{Machine Translation.}-системы);
		\item перевод, осуществляемый человеком, 
			с использованием компьютера (TM\footnote{Translation memory.}-системы).
	\end{itemize}
	
	Полностью автоматические системы машинного перевода являются 
	несбыточной мечтой, чем реальной идей. 
	В этой работе мы их~рассматривать не будем. 
	Все системы машинного перевода (MT-системы) работают при участии человека 
	в той или иной мере. 
	TM-системы иногда называют еще <<памятью переводчика>>. 
	Они  являются скорее просто удобным инструментом, 
	нежели элементом автоматизации.

