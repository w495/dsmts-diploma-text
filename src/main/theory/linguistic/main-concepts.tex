%%%%%%%%%%%%%%%%%%%%%%%%%%%%%%%%%%%%%%%%%%%%%%%%%%%%%%%%%%%%%%%%%%%%%%%%%%%%%%%%
%%%
%%% CONCEPTS
%%%

\subsubsection{Подходы к машинному переводу}

Системы машинного перевода могут использовать метод перевода основанный 
на~лингвистических правилах. 
Наиболее подходящие слова из исходного языка просто 
заменяются словами переводного языка.
Часто утверждается, что для успешного решения проблемы машинного перевода 
необходимо решить проблему понимания текста на естественном языке.

Как правило, метод перевода основанный на правилах использует 
символическое представление (посредника), на основе которого создается 
текст на переводном языке. А если учитывать природу посредника 
то~можно говорить об интерлингвистическом машинном 
переводе или~трансферном машинном переводе. 
Эти методы требуют очень больших словарей с морфологической, 
синтаксической и семантической информацией и большого набора правил. 
Современные системы машинного перевода делят на три большие группы:
\begin{itemize}
	\item  основанные на правилах;
	\item  основанные на примерах;
	\item  статистические.
\end{itemize}

\pagebreak

\subsubsection{СМП основанные на~правилах}
Системы машинного перевода основанные на правилах --- общий термин, 
который обозначает системы машинного перевода 
на основе лингвистической информации об исходном и переводном языках. 
Они состоят из двуязычных словарей и грамматик, 
охватывающих основные семантические, морфологические, 
синтаксические закономерности каждого языка. 
Такой подход к машинному переводу еще~называют классическим.
На основе этих данных исходный текст последовательно, 
по~предложениям, преобразуется в текст перевода. 
Эти системы противопоставляют системам машинного перевода, 
которые основаны на~примерах.
Принцип работы таких систем --- 
связь структуры входного и выходного предложения. 

Эти системы делятся на три группы:
\begin{itemize}
	\item системы пословного перевода;
	\item трансферные системы;
	\item интерлингвистические.
\end{itemize}

\paragraph{Пословный перевод}
Такие системы используются сейчас крайне редко из-за низкого качества перевода. 
Слова исходного текста преобразуются как есть в слова переводного текста. 
Часто такое преобразование происходит без~лемматизации и морфологического анализа. 
Это самый простой метод машинного перевода. 
Он используется для перевода длинных списков слов, например, каталогов. 
Так же он может быть использован 
для составления <<словаря-подстрочечника>> для TM-систем.

\paragraph{Трансферные системы}
Как трансферные системы, так и интерлингвистические, 
имеют одну и~ту~же общую идею. 
Для перевода необходимо иметь посредника, 
который в~себе несет смысл переводимого выражения. 
В~интерлингвистических системах посредник 
не~зависит от~пары языков, в~то время как в~трансферных --- зависит. 

Трансферные системы работают по очень простому принципу: 
к~входному тексту применяются правила, которые ставят 
в~соответствие структуры исходного и переводного языков. 
Начальный этап работы включает в себя морфологический, 
синтаксический, а иногда и~семантический анализ 
текста для создания внутреннего представления. 
Перевод генерируется из этого представления 
с~использованием двуязычных словарей и~грамматических правил. 
Иногда на основе первичного представления, 
которое было  получено из исходного текста, 
строят более <<абстрактное>> внутренне представление. 
Это делается для~того, чтобы акцентировать места важные для перевода 
и~отбросить несущественные части текста. 
При построении текста перевода преобразование уровней внутренних 
представлений происходит в~обратном порядке.

При использовании этой стратегии получается достаточно 
высокое качество переводов, 
с точностью в районе 90\% (сильно зависит от~языковой пары). 
Работа любой системы трансферного перевода состоит как минимум из пяти частей:
\begin{itemize}
	\item  морфологический анализ;
	\item  лексическая категоризация;
	\item  лексический трансфер;
	\item  структурный трансфер;
	\item  морфологическая генерация.
\end{itemize}

\paragraph{Интерлингвистический машинный перевод}

Интерлингвистический машинный перевод --- один из классических подходов 
к~машинному переводу. 
Исходный текст трансформируется в~абстрактное представление, 
которое не зависит от языка (в отличие от~трансферного перевода). 
Переводной текст создается на основе этого представления. 
Можно доказать математически, что в рамках этого подхода, 
создание каждого нового интерпретатора языка для такой системы будет удешевлять ее, 
по~сравнению, например, с системой трансферного перевода. 
Кроме того, в рамках такого подхода можно реализовать <<пересказ~текста>>, 
перефразирование исходного текста в~рамках одного языка.

До сих пор не существует реализаций такого типа систем, 
которая корректно работала хотя бы для двух языков. 
Многие эксперты высказывают сомнения в возможности реализации. 
Самая большая сложность для создания подобных систем заключается 
в проектировании межъязыкового представления. 
Оно должно быть одновременно абстрактным и независящим от конкретных языков, 
но в тоже время оно должно отражать особенности любого существующего языка. 
С другой стороны, в рамках искусственного интеллекта, 
\emph{задача выделения смысла} текста на данный момент до сих пор не решена.


\subsubsection{СМП основанные на примерах}

Перевод основанный на примерах --- один из~подходов 
к~машинному переводу, при котором используется 
двуязычный корпус текста. 
Этот корпус текста во время перевода используется как база знаний. 
Предполагается, что люди разлагают исходный текст на фразы, 
потом переводят эти~фразы, 
а далее составляют переводной текст из фраз. 
Причем, перевод фраз обычно происходит по аналогии с предыдущими переводами.
Для построения системы машинного перевода, основанной на примерах  
потребуется языковой корпус, составленный из пар предложений.
\emph{Языковые пары} --- тексты, содержащие предложения на одном 
языке и~соответствующие им предложения на втором, могут быть 
как вариантами написания двух предложений 
человеком --- носителем двух языков, 
так и набором предложений и их переводов, выполненных человеком.

Перевод, основанный на примерах, лучше всего подходит 
для таких явлений как фразовые глаголы. 
Значения фразовых глаголов сильно зависит от контекста. 
Фразовые глаголы очень часто встречаются 
в~разговорном английском языке. 
Они состоят из глагола с предлогом или наречием. 
Смысл такого выражения невозможно получить 
из~смыслов составляющих частей. 
Классические методы перевода в данном случае неприменимы. 
Этот метод перевода можно использовать 
для~определения контекста предложений.

Как показано далее, реализовать примитивную систему 
машинного перевода основанную на примерах крайне просто.

\subsubsection{Статистический машинный перевод}

Статистический машинный перевод --- это метод машинного перевода.
Он использует сравнение больших объемов языковых пар, 
так же как и~машинный перевод основанный на примерах. 
Статистический машинный перевод обладает свойством <<самообучения>>. 
Чем больше в распоряжении имеется языковых пар 
и чем точнее они соответствуют друг другу, 
тем~лучше результат статистического машинного перевода. 
Статистический машинный перевод основан на поиске 
наиболее вероятного перевода предложения с~использованием 
данных из~двуязычных корпусов текстов. 
В результате при выполнении перевода компьютер 
не оперирует лингвистическими алгоритмами, 
а вычисляет вероятность применения того или иного слова или выражения. 
Слово или~последовательность слов, 
имеющие оптимальную вероятность, 
считаются наиболее соответствующими переводу исходного 
текста и~подставляются компьютером в получаемый в результате текст. 

В статистическом машинном переводе ставится 
задача не перевода текста, а задача его расшифровки. 
Статья, написанная на~английском языке, 
на самом деле является статьей написанной на~русском, 
но текст зашифрован (или искажен шумом). 
При таком подходе становится понятно почему, 
чем <<дальше>> языки, тем лучше работает статистический метод, 
по сравнению с классическими подходами.

\paragraph{Модель Шеннона}

\begin{figured}
	\tikzstyle{rootf} = [draw=yellow!50!black!70,thick,minimum height=1cm,minimum width=1.5cm,top color=yellow!20,bottom color=yellow!60!black!20,decorate,decoration={random steps,segment length=3pt,amplitude=1pt}]
	\tikzstyle{pointf} = [rounded rectangle,thick,minimum size=1cm,draw=blue!50!black!50,top color=white,bottom color=blue!50!black!20]
	\tikzstyle{itemf} = [rectangle,thick,minimum width=1.5cm,minimum height=1cm,draw=teal!50!black!50,top color=white,bottom color=teal!50!black!20]
	\begin{tikzpicture}[thick, node distance=4cm, text height=0.9ex, text depth=.25ex, auto]
		\node[pointf] (source-point) {\small Передачик};
		\node[pointf,right of=source-point] (target-point) {\small Приемник};	
		\node[itemf,left of=source-point] (source) {\small Источник ($R$)};
		\node[itemf,right of=target-point] (target) {\small Цель ($E$)};	
			\path[->, black] (source) edge (source-point);
			\path [->, red, decoration={zigzag,segment length=5,amplitude=2,
				post=lineto,post length=2pt},font=\scriptsize,line join=round] 
					(source-point) edge[decorate] node[auto] {Шум} (target-point);
			\path[->, black] (target-point) edge (target);
	\end{tikzpicture}
	\fcaption{Модель зашумленного канала.}
\end{figured}


Модель состоит из пяти элементов: источника информации, 
передатчика, канала передачи, приемника и конечной цели, расположенных линейно.
Передатчик кодирует информацию, полученную от источника, 
и передает ее на канал. По каналу передачи, на который действует 
шум — помехи любого рода, искажающие информацию, данные поступают в приемник, 
где они декодируется и передаются к конечной цели. 

Из-за шума полученная приемником информация в общем случае не~совпадает с информацией, 
отправленной передатчиком. Однако, согласно модели Шеннона, 
создавая избыточную информацию, исходные данные можно восстановить 
со~сколь угодно высокой вероятностью. 
Для~обнаружения ошибок используются контрольные суммы, 
для их~исправления — специальные корректирующие коды,
при~условии, что~степень шума не превосходит некоторой границы.
Стоит отметить, что любая информация в некотором роде избыточна \cite{Shannon:1948}. 
Человеческая речь избыточна --- чтобы уловить смысл предложения, 
зачастую необязательно слышать его полностью. 
Аналогично, письменная речь, тоже избыточна, 
и при переводе этим можно воспользоваться. 
Если предложение в целом понятно, но есть несколько незнакомых слов, 
то обычно не трудно догадаться об~их~значении. 

Таким образом, для перевода текста необходимо найти способ декодирования, 
использующий естественную избыточность, 
в связи с чем декодирование должно быть вероятностным. 
Задача такого декодирования заключается в том, чтобы, при данном сообщении 
найти исходное сообщение, которому соответствует наибольшая вероятность. 
Для этого же необходимо для любых двух сообщений уметь находить 
условную вероятность того, что переведенное сообщение, 
пройдя через канал с шумом, преобразуется в исходное сообщение.
В данном случае нужна модель источника (модель языка) 
и модель канала (модель перевода). 
Модель языка дает оценку вероятности фразам переводного языка, 
а модель перевода оценивает вероятность 
исходной фразы при условии фразы на переводном языке.
Если нам нужно перевести фразу с русского на английский, 
то мы должны знать, что именно обычно говорят по-английски 
и как~английские фразы искажаются до состояния русского языка. 
Сам по себе перевод превращается 
в~процесс поиска такой английской фразы, 
которая максимизировала бы произведения безусловной вероятности английской 
фразы и вероятности русской фразы-оригинала при условии данной английской фразы.
\[
	\max\limits_{\FE} P(\FE|\FR) =  \max\limits_{\FE} \left(  P(\FE) \cdot P(\FR|\FE) \right), \text{ где}
\]
\begin{itemize}
	\item  $\FE$ --- фраза перевода (английская);
	\item  $\FE$ --- фраза оригинала (русская).
\end{itemize}

В системах статистического перевода, в качестве модели языка используются варианты 
$n$-граммной модели (например, в переводчике Google, использутеся $5$-граммная модель). 
Согласно этой модели, правильность выбора того или иного слова зависит 
только от~предшествующих $(n-1)$ слов. 
Самой простой статистической моделью перевода является модель пословного перевода. 
В этой модели, известной как Модель IBM №1, предполагается, 
что для перевода предложения с одного языка 
на~другой достаточно перевести все слова, 
а расстановку их~в~правильном порядке обеспечит модель языка. 
Единственным массивом данных, которым оперирует Модель №1, 
является таблица вероятностей парных переводных соответствий 
слов двух языков \cite{Рахимбердиев:2003}. Обычно используются 
более сложные модели перевода. 

Работа статистических систем, 
так же как и систем основанных на~примерах происходит 
в двух режимах: обучения и эксплуатации. 
В режиме обучения просматриваются параллельные корпуса 
текста и~вычисляются вероятности переводных соответствий. 
Строится модель языка перевода. 
Тут же определяются вероятности каждой $n$-граммы.
В режиме эксплуатации, для фразы из~исходного текста 
ищется фраза переводного текста, так, 
чтобы максимизировать произведение вероятностей.

\additionaltext{
\subsection{ТМ-системы}
	После работы СМП (трансферного типа, Example-Based) 
	не опознанные фрагменты текста переводятся на иностранный язык вручную. 
	При этом можно воспользоваться процедурой приближенного поиска 
	этих фрагментов в базе данных, 
	а результаты поиска использовать как~подсказку. 
	Результаты ручного перевода новых фрагментов 
	текстов можно снова вводить в базу данных. 
	Тогда, по мере перевода все новых и~новых документов, 
	<<память переводчика>> будет постепенно обогащаться, 
	и ее эффективность будет возрастать. 
	Бесспорным достоинством технологии <<памяти переводчика>> 
	является высокое качество перевода того класса текстов, 
	для которого она создавалась.
	Но базы переводных соответствий, построенные для однородных 
	текстов одного предприятия, пригодны лишь для однородных 
	текстов близких по~профилю предприятий, 
	так как предложения и большие фрагменты предложений, 
	извлекаемые из текстов одних документов, как правило, 
	не~встречаются или очень редко встречаются 
	в~текстах других документов. 
	Практическая реализация связаны с большими трудозатратами 
	на создание <<памяти переводчика>> 
	или пополнение массивов двуязычных текстов (билингв). 
	По такой системе чаще всего и переводятся научные, 
	технические и математические тексты. 
	Авторам этой работы, в частности, известно, 
	что подобный подход часто используется Курчатовском институте.
}
