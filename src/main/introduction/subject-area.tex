%%%%%%%%%%%%%%%%%%%%%%%%%%%%%%%%%%%%%%%%%%%%%%%%%%%%%%%%%%%%%%%%%%%%%%%%%%%%%%%%
%%%
%%% AREA
%%%
	
	Мы живем в мире информационных технологий, которые прочно вошли 
	в нашу жизнь. Мы пользуемся современными средствами связи. 
	Компьютер превратился в неотъемлемый элемент нашей жизни не только 
	на рабочем месте, но и в повседневной жизни. Быстрое развитие новых 
	информационных технологий свидетельствует о всевозрастающей роли 
	компьютерной техники в мировом информационном пространстве. 
	
	С каждым днем увеличивается число пользователей Интернета. Все 
	больше сетевые технологии оказывают влияние на развитие самой науки 
	и техники. За последние годы сильно начал меняться характер 
	образования, переходя на уровень дистанционного. Этот переход 
	осуществляется даже в классических вузах. Развитие науки и образования, 
	да и вообще формирование мирового информационного пространства 
	значительно тормозится из-за так называемого языкового барьера. 
	Эта проблема пока не нашла своего кардинального решения. 
	
	Последние годы объем предназначенной для перевода информации 
	увеличился. Создание универсального языка типа Эсперанто, <<эльфийских 
	языков>> или какого-либо другого языка не привели к изменению ситуации. 
	Использование традиционных средств межкультурной коммуникации 
	может быть достойным выходом. Нынешний век диктует свои условия: 
	информация меняется двадцать четыре часа в сутки, широко применяются 
	электронные средства связи. В такой ситуации классический подход 
	к осуществлению перевода не всегда оправдывает себя. Он требует 
	значительных капиталовложений и временных затрат. В некоторых 
	случаях более целесообразным представляется использование машинного 
	или автоматического перевода. 
