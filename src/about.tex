%%%%%%%%%%%%%%%%%%%%%%%%%%%%%%%%%%%%%%%%%%%%%%%%%%%%%%%%%%%%%%%%%%%%%%%%%%%%%%%%
%%%
%%% ABOUT
%%%

\section*{От автора}

\begin{flushright}
	{\magic Текущая сборка документа: \today \ \thistime \\}
\end{flushright}
Техника в жизни человека играет значительную роль 
и машинный перевод не~стал исключением. 
Для облегчения и~упрощения перевода 
была сделана попытка разработать 
распределенную статистическую систему машинного перевода.

Работа сверстана c использованием {\comic \XeLaTeX}.
Документ оформлен по требованиям Московского Авиационного Института,
не всегда эти требования соответствуют принятым государственным или мировым стандартам,
да и~здравому смыслу. Так что, уважаемые читатели, --- не~взыщите.

Спасибо Дмитрию Кану ($\rightsquigarrow$ \cite{Кан:2011}), за исправление опечаток предыдущей сборки.
При~обнаружении опечаток и~ошибок, пожалуйста, обращайтесь по~адресу: 
\href{mailto:w@w-495.ru}{{\color{teal} w@w-495.ru}}.\\

\begin{center}
	\vspace{12pt}
	\begin{tabular}{p{7cm}}
		\hline \\
	\end{tabular}\\
	\vspace{12pt}
\end{center}
{\libertine
    Версия документа собрана специально для
    %Ольги~Игоревны~Денисовой, \\ кандидата филологических наук, доцента кафедры И-$01$ ИИЯ МАИ.
    %Евгения~Сергеевича~Гаврилова, ассистента кафедры 806 МАИ.
    %Ольги~Владимировны~Захаровой, кандидата экономических наук, преподавателя кафедры Экономической Кибернетики факультета Экономической Информатики Харьковского Национального Экономического Университета.
    %ресурса \href{http://www.twirpx.com}{twirpx.com}
    ресурса \href{http://www.slideshare.net/}{slideshare.net}
    %ресурса \href{http://mai.academia.edu}{academia.edu}

}

\pagebreak

