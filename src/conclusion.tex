%%%%%%%%%%%%%%%%%%%%%%%%%%%%%%%%%%%%%%%%%%%%%%%%%%%%%%%%%%%%%%%%%%%%%%%%%%%%%%%%
%%%
%%% Заключение
%%%

\Csection{Заключение}

В ходе выполнения работы были получены следующие результаты:

{\renewcommand{\labelenumi}{\alph{enumi})}
	\begin{enumerate}
		\item реализован основной функционал распределенной системы машинного перевода;
		\item разработаны методы создания модели машинного перевода текстов научно-технической литературы;
		\item реализован эффективный метод декодирования (перевода) с~одного языка на~другой;
		\item была проверена модель параллельных вычислений для задачи машинного перевода
			и сделаны выводы о~приросте производительности на~ряде операций. 
		\item выяснилось, что используемая модель вычислений не экономична 
			по отношению к~базе данных;
		\item не получилось достичь высокого качества перевода для текстов общей тематики, 
			используемые алгоритмы могут потребовать корректировки.
	\end{enumerate}
}
В качестве дальнейших приоритетных направлений исследований 
и развития созданной системы можно выделить следующие:
{\renewcommand{\labelenumi}{\alph{enumi})}
	\begin{enumerate}
		\item реализация полноценного фразового перевода
			при использовании классических подходов 
				к статическому машинному переводу совместно с~подходом рассмотренном в~работе;
		\item реализация синтаксического перевода;
		\item апробация смешанной трансферно-статистической системы машинного перевода 
			при использовании морфологического анализа для конкретной пары языков (русский, английский);
			% 
			% Существует не мало различныех средств 
			% для морфологического анализа текста.===
			% Например:
			% 	Готовые решения:
			% 		lemmatizer -> http://lemmatizer.org/
			% 		mystem -> http://company.yandex.ru/technology/mystem/
			% 	Библиотеки:
			% 		Морфологический анализатор pymorphy
			% 		Python NLTK
			% 		Java Apache NLTK
			% 
		\item апробация более точных, но менее эффективных методов поиска.
	\end{enumerate}
}
Кроме того, в~плане улучшения реализации системы может потребоваться
{\renewcommand{\labelenumi}{\alph{enumi})}
	\begin{enumerate}
		\item использовать пословное сжатие при хранении в~БД;
		\item переписать обработчика на~C с~библиотекой libevent;
		\item использовать libevent для rest-интерфейса декодера,
			чтобы иметь возможность поддерживать 1 млн. одновременных соединений;
		\item попробовать перейти на~более эффективную реализацию базы данных, 
			например с~redis на~leveldb.
	\end{enumerate}
}


\pagebreak




