%%%%%%%%%%%%%%%%%%%%%%%%%%%%%%%%%%%%%%%%%%%%%%%%%%%%%%%%%%%%%%%%%%%%%%%%%%%%%%%%
%%%
%%% Заключение
%%%

\Csection{Заключение}

В ходе выполнения дипломной работы были получены следующие результаты:

{\renewcommand{\labelenumi}{\alph{enumi})}
	\begin{enumerate}
		\item реализован основной функционал распределенной системы машинного перевода;
		\item разработаны методы создания модели машинного перевода текстов научно-технической литературы;
		\item реализован эффективный метод декодирования (перевода) с~одного языка на~другой;
		\item была проверена модель параллельных вычислений для задачи машинного перевода
			и сделаны выводы о~приросте производительности на~ряде операций. 
		\item выяснилось, что используемая модель вычислений не экономична 
			по отношению к~базе данных;
		\item не получилось достичь высокого качества перевода для текстов общей тематики, 
			используемые алгоритмы могут потребовать корректировки.
	\end{enumerate}
}
В качестве дальнейших приоритетных направлений исследований 
и развития созданной системы можно выделить следующие:
{\renewcommand{\labelenumi}{\alph{enumi})}
	\begin{enumerate}
		\item реализация полноценного фразового перевода
			при использовании классических подходов 
				к статическому машинному переводу совместно с~подходом рассмотренном в~работе;
		\item реализация синтаксического перевода;
		\item апробация смешанной трансферно-статистической системы машинного перевода 
			при использовании морфологического анализа для конкретной пары языков (русский, английский);
			% 
			% Существует не мало различныех средств 
			% для морфологического анализа текста.===
			% Например:
			% 	Готовые решения:
			% 		lemmatizer -> http://lemmatizer.org/
			% 		mystem -> http://company.yandex.ru/technology/mystem/
			% 	Библиотеки:
			% 		Морфологический анализатор pymorphy
			% 		Python NLTK
			% 		Java Apache NLTK
			% 
		\item апробация более точных, но менее эффективных методов поиска.
	\end{enumerate}
}
Кроме того, в~плане улучшения реализации системы может потребоваться
{\renewcommand{\labelenumi}{\alph{enumi})}
	\begin{enumerate}
		\item использовать пословное сжатие при хранении в~БД;
		\item переписать обработчика на~C с~библиотекой libevent;
		\item использовать libevent для rest-интерфейса декодера,
			чтобы иметь возможность поддерживать 1 млн. одновременных соединений;
		\item попробовать перейти на~более эффективную реализацию базы данных, 
			например с~redis на~leveldb.
	\end{enumerate}
}

В экономической части работы проведена оценка стоимости 
разрабатываемой системы.
Суммарные расходы на~разработку могут составлять. $916669.08$ рублей.
Однако, чтобы система окупилась достаточно продавать ее по цене $1833$ рублей.
Это связано, с~большим числом предприятий нуждающихся в~оперативном машинном переводе.
Совокупная стоимость владения системой при работе в~режиме $24$ на~$7$ на~$365$ составляет
108786 рублей в~год. Однако, это все равно ниже чем расходы на~заработную плату
низкоквалифицированного переводчика за тот же самый период.
Даже при наличие корректора, система оказывается очень выгодной.

В разделе посвященном охране труда и~окружающей среды 
кратко описаны основные требования к~рабочему месту
программиста. Эти же самые рекомендации можно отнести 
и~к~рабочему месту профессионального переводчика

Наиболее подробно в~работе рассмотрен фактор зашумленности рабочего места. 
В современном мире переводчик чаще всего использует в~своей работе
вычислительную технику.
Как составление программ, так и~перевод текстов 
--- сложная высокоинтеллектуальная деятельность,
часто сопряженная с~высокими нервными нагрузками. 
При работе с~любой техникой необходимо соблюдать спокойствие, 
взвешенно и~обдумано принимать решения, 
и не поддаваться внешним раздражителям.
Одним из таких раздражителей является шум. 
Существуют различные методы защиты от шума.
Чаще всего это звукоизоляция помещений, 
специальные установки для устройств, поглощающие вибрацию. 
В некоторых случаях применяют малошумящие устройства.

Соблюдение рекомендаций, предложенных выше, позволит свести к~минимуму
вредные воздействия на~здоровье человека 
при длительной работе и~сохранить его работоспособность, 
a с~помощью последнего повысить качество результатов его труда.

Кроме того, системы машинного перевода призваны ограничить взаимодействие 
переводчиков с~вычислительной техникой, и~тем самым так же свести к~минимуму
вредные воздействия на~их здоровье.
Это позволит улучшить качество переводов, 
на которых будет обучаться система в~будущем.

\pagebreak




