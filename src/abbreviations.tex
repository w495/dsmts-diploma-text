%%%%%%%%%%%%%%%%%%%%%%%%%%%%%%%%%%%%%%%%%%%%%%%%%%%%%%%%%%%%%%%%%%%%%%%%%%%%%%%%
%%%
%%% Расчет затрат на разработку
%%%

\Csection{Список терминов и их сокращений}

\begin{description}
	\item[CMП] --- система машинного перевода.
	\item[CCMП] --- статистическая система машинного перевода.
	\item[Корпус (лингвистический корпус)] --- называют совокупность текстов, собранных в соответствии с определенными принципами, размеченных по определенному стандарту, в этой работе корпусом называют совокупность предложений на конкретном языке,
	разделенных символом перевода строки.
	\item[n-грамма] --- подпоследовательность из $n$ элементов из данной последовательности текста или речи, в данной работе рассматривается как подпоследовательность слов.
	\item[ОТП (OTP)] --- открытая телекоммуникационная платформа (open telecom platform).
	\item[Cупервизор (в терминах ОТП)] --- процесс (как совокупность взаимосвязанных и взаимодействующих действий),
		следящий за дочерними процессами, отвечающий за их запуск и остановку.
	\item[Приложение (в терминах ОТП)] --- компонент, который можно запускать и останавливать как единое целое, и который также может быть использован повторно в других системах. 
	\item[Дерево контроля (супервизии)] --- совокупность рабочих процессов вычислительной системы, их супервизоров представленное в виде древовидной структуры.
\end{description}

\pagebreak

